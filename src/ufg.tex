\section{Common presentation elements}

\subsection{Box}

\setLayout{vertical}
\begin{frame}{Example on using box}

    \footnotesize
    
    \begin{ex}
        Em uma versão da linguagem BASIC, o nome de uma variável é uma sequência de um ou dois caracteres alfanuméricos, em que letras maiúsculas e minúsculas não são distinguidas. Além disso, um nome de variável deve começar com uma letra e deve ser diferente das cinco sequências de dois caracteres reservadas para o uso de comandos. Quantos nomes diferentes de variáveis são possíveis nesta versão do BASIC?
    \end{ex}
    
    \begin{block}{Solução}
        Pela regra da soma, $V=V_1+V_2$. Como as variáveis só podem começar com letras, temos que $V_1=26$. Pela regra do produto, há $26\cdot 36=936$ sequências de tamanho $2$ que comecem com uma letra e terminam com um caracter alfanumérico. Porém, não se deve usar $5$ variáveis reservadas. Assim, $V_2=26\cdot 36-5=931$. Logo, há $V=V_1+V_2 = 26+931=957$ nomes diferentes para variáveis nesta versão do BASIC.
    \end{block}

\end{frame}
%---------------------------------------------------------


%--------------------------------------------------------- Slide 3
\subsection{Table}

\begin{frame}{Example on using table}

    \begin{table}[]
        \centering
        \caption{\label{tab:1}Countries and their codes}
        
        \renewcommand{\arraystretch}{1.5}
        \setlength{\tabcolsep}{10pt}
        
        {\rowcolors{2}{}{LightGray!10}
            \begin{tabular}{ p{3cm}p{3cm}p{3cm}  }
                \toprule 
                \textbf{Country Name} & \textbf{Code 2} & \textbf{Code 3} \\
                \midrule
                Afghanistan & AF &AFG \\
                Aland Islands & AX   & ALA \\
                Albania &AL & ALB \\
                Algeria    &DZ & DZA \\
                \bottomrule
            \end{tabular}
        }
    \end{table}
    
\end{frame}
%---------------------------------------------------------


%---------------------------------------------------------


%--------------------------------------------------------- Slide 5
\section{Changing colors and Layouts}

\setLayout{blank} % Example of changing layout
\setBGColor{DarkOrange}  %Example of changing background color 

\begin{frame}{Clean layout and two-column text}
    
    \begin{columns}
    
        \column{0.5\textwidth}
        This is a text in first column.
        $$E=mc^2$$
        $$ 1 + 2 + \cdots + k =  \frac{k \cdot (k + 1)}{2}.$$
        \begin{itemize}
        \item First item
       
        \item Second item
        \end{itemize}
        
        \column{0.5\textwidth}
        This text will be in the second column
        and on a second tought this is a nice looking
        layout in some cases.
        
        \begin{enumerate}
            \item First
            \item Second
        \end{enumerate}
        
    \end{columns}
    
\end{frame}
%---------------------------------------------------------


%---------------------------------------------------------Slide 6
%Highlighting text
\setLayout{vertical}
\begin{frame}{Sample frame title}
    
    In this slide, some important text will be
    \alert{highlighted} because it's important. Please, don't abuse it.
    
    \begin{block}{Remark}
        Sample text
    \end{block}
    
    \begin{alertblock}{Important theorem}
        Sample text in alert box
    \end{alertblock}
    
    \begin{examples}
        Sample text in green box. The title of the block is ``Examples".
    \end{examples}
    
\end{frame}



%---------------------------------------------------------
