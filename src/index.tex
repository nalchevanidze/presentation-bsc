
\section{Motivation}

\begin{frame}{}
    \frametitle{Motivation}

    Heute werden riesige Mengen an Informationen über das Web ausgetauscht. Dies erhöht die Komplexität der Anwendungen. Morpheus GraphQL versucht, diese Komplexität mit der Funktionssprache Haskell und GraphQL zu reduzieren, indem es Flexibilität und Typsicherheit bietet, indem es eine direkte api-Beschreibungsmethode mit regulären Haskell-Typen bereitstellt. 

\end{frame}

\setLayout{Zielsetzung}
\begin{frame}
    \frametitle{Zielsetzung}

    Für die Arbeit werden die folgenden Ziele gesetzt.    

    \begin{itemize}

        \item \textbf{Sicherheit und Ausdrucksstärke}: Wir werden versuchen, Laufzeitfehler zu vermeiden und es Entwicklern zu ermöglichen, komplexe Fälle im Code auszudrücken.
        
        \item \textbf{Flexibilität und Leistung} : wir werden  versuchen, verschiedene Nutzungskontexte (Gärete) anzusprechen, aber dennoch performant zu bleiben. 
        
        \item \textbf{Wartbarkeit und Benutzerfreundlichkeit}
       :  Die API muss für Dritte leicht zu verstehen und zu verwenden sein, und die Bibliothek muss einen unkomplizierten Ansatz für die API-Definition bieten. 

    \end{itemize}   
\end{frame}

\section{Haskell}

\setLayout{mainpoint}
\begin{frame}{}
    \frametitle{Haskell}
\end{frame}


\section{GraphQL}
\setLayout{mainpoint}
% \setBGColor{BGDarkRed}
\begin{frame}{}
    \frametitle{GraphQL}
\end{frame}


\subsection{Morpheus GraphQL}
\setBGColor{white}
\begin{frame}{Morpheus GraphQL}
    \begin{figure}
        \centering
        \includegraphics[width=1.1\textwidth]{assets/img/morpheus-graphql-bg.png}
    \end{figure}
\end{frame}