%!TEX root = ../../main.tex

\section{Design Decisions} 

\begin{frame}[allowframebreaks]\frametitle{Design Decisions}
\begin{block}{Code-First Approach}  

This approach generates the schema and validates the resolver types simultaneously, whereby a single type change automatically updates the resolver types and the GraphQL schema, making API maintenance easier.

\end{block}

\begin{block}{Embeded Domain Specific Language Based on Datatype-Generic Programming}  

native language syntax with domain-specific semantics, users can focus on domain-specific problems~\cite{edsl-modeling}. futhermore increases familiarity. datatype-generic programming eliminates boilerplate code by encouraging the programmer to avoid implementing tedious and high-maintenance boilerplate code~\cite{scrap-your-boilerplate}.

\end{block}

\begin{block}{Monadic Resolvers} 

Since all side effects and behavioral extensions in Haskell are performed with monads, field resolvers will be monadic functions. 
\end{block}


\begin{block}{Parameterized Resolver Types}

parametric polymorphism enablefinition of resolver types independent of their operations, where the type parameter determines the allowed operations.  

\expr{data Type m = Type \{ resolver :: m Value\}}

the type \expr{Type Q} constructs resolver \expr{resolver :: Q Value} with query operations. 
 
the type \expr{Type M} constructs resolver \expr{resolver :: M Value} with mutation operations.

\end{block}
\end{frame}