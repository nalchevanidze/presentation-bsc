%!TEX root = ../../main.tex

\section{Grundlage}

%\setLayout{vertical}
\begin{frame}\frametitle{Haskell}

    \footnotesize
    \begin{block}{Haskell}
        Eine \alert{nicht strenge}, \alert{rein funktionale} Sprache mit statischer 
        Typisierung und algebraischen Datentypen \cite{history-of-haskell}.
    \end{block}

    \begin{block}{Merkmale}
        eine reichhaltige Sprache, die nützliche Funktionen wie 
        Currying, Infix-Präfix-Operationen, anonyme Funktionen und
        Listenverständnis, Monaden, 
        generische Programmierung, 
        Template-Meta-Programmierung 
        ..usw. bietet \cite{history-of-haskell}.
    \end{block}

    \begin{block}{Haskell ist schön}
        Viele Entwickler behaupten, dass Haskell-Programme schön aussehen. 
        \cite{history-of-haskell} 
    \end{block}

\end{frame}


% \setLayout{vertical}
\begin{frame}{GraphQL}

    \footnotesize

    \begin{block}{Was ist GraphQL?}
        GraphQL ist eine API-Abfragesprache zur Lösung der Effizienzprobleme der Kommunikation\cite{gql-iot}.         
    \end{block}

    \begin{block}{von Facebook Entwicklt}
        Es wurde drei Jahre lang intern bei Facebook entwickelt und seine Spezifikation und seine Referenzimplementierung 2016 veröffentlicht.
        \cite{initial-analysis-of-gql}
    \end{block}

    \begin{block}{GraphQL als aktueller Trend}
        Seit ihrem ersten Erscheinen hat sie ein reiches Open-Source-Ökosystem und das Vertrauen von Unternehmen aus verschiedenen Sektoren gewonnen. z.B: (GitHub), Unterhaltung (Netflix), Finanzen (PayPal), Reisen (KLM), etc \cite{morph-gql-1,gql-healthcare}.
    \end{block}

\end{frame}


% \section{Morpheus GraphQL}
% \setBGColor{white}
% \begin{frame}{Morpheus GraphQL}
%     \begin{figure}
%         \centering
%         \includegraphics[width=1.1\textwidth]{assets/img/morpheus-graphql-bg.png}
%     \end{figure}
% \end{frame}