%!TEX root = ../../main.tex

\section{Theoretische Grundlage}

\subsection{Haskell}
\setLayout{mainpoint}
\begin{frame}{}
    \frametitle{Haskell}
\end{frame}

\setLayout{vertical}
\begin{frame}{}

    \begin{alertblock}{Haskell}
        Haskell ist eine nicht strenge, rein funktionale Sprache mit statischer Typisierung und algebraischen Datentypen. Es ist eine reichhaltige Sprache, die nützliche Funktionen wie Currying, Infix-Präfix-Operationen, anonyme Funktionen und Listenverständnis, Monaden, bietet. 
    \end{alertblock}

\end{frame}


\subsection{GraphQL}
\setLayout{mainpoint}
% \setBGColor{UHHRed}
\begin{frame}{}
    \frametitle{GraphQL}
\end{frame}


\section{Morpheus GraphQL}
\setBGColor{white}
\begin{frame}{Morpheus GraphQL}
    \begin{figure}
        \centering
        \includegraphics[width=1.1\textwidth]{assets/img/morpheus-graphql-bg.png}
    \end{figure}
\end{frame}