\section{Motivation}

\begin{frame}\frametitle{Motivation}

    \footnotesize

    \begin{alertblock}{Aktuelle Herausforderungen}
        Heute werden riesige Mengen an Informationen über das Web ausgetauscht. Dies erhöht die \alert{Wartezeit} und die \alert{Komplexität} der Anwendungen. 
    \end{alertblock}

    \begin{alertblock}{Haskell und  GraphQL}
        Die Funktionssprache Haskell und GraphQL können diese Probleme verringern.
    \end{alertblock}

    \begin{block}{Fehlende Bibliothek}
        Es gibt einige Bibliotheken, die GraphQL in Haskell implementieren, aber entweder bieten sie keine ausreichende Typsicherheit oder sie bilden GraphQL in Haskell auf sehr komplizierte Weise ab. 
    \end{block}

\end{frame}

\begin{frame}\frametitle{Zielsetzung}

    In dieser arbeit versuchen wir, eine Bibliothek bereitzustellen,
    die \alert{typensicherheit bietet} und dennoch \alert{einfach zu schreiben} ist. 
    dabei sollen folgende ziele verfolgt werden

\end{frame}