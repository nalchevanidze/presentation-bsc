\section{Introduction}

\begin{frame}\frametitle{Motivation}

    \footnotesize

    \begin{alertblock}{Current Challenges}
        Modern web applications continuously transfer vast amounts of data between clients and the server. Various applications have a rapid development cycle and need to simultaneously support multiple devices, presenting complexity, performance, and reliability challenges in development. 
    \end{alertblock}

    \begin{alertblock}{Haskell and  GraphQL}
        Haskell has proven to be safe and capable of managing the complexity. GraphQL enables clients to query only the data they need to improve performance, while servers can address multiple devices. Their combination can be useful to solve other problems.
    \end{alertblock}

    \begin{block}{Fehlende Bibliothek}
        Existing GraphQL Haskell libraries either do not provide sufficient type safety or straightforward mapping, making them cumbersome to use.
    \end{block}

\end{frame}

\begin{frame}\frametitle{Objectives}

We should welcome newcomers and offer them straightforward access instead of discouraging them by striving for extensibility at the type-level.  Our goal is to provide a library that requires minimal effort to use and provides a good developer experience. To achieve this, we will implement a Haskell GraphQL library that can build reliable, easy-to-use, maintainable, and efficient GraphQL APIs.

\end{frame}