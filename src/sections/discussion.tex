\section{Discussion}

    
\begin{frame}\frametitle{Advantages}

\begin{itemize}
    \item easy-to-use interface while ensuring type safety.
    \item The library uses an intuitive design that can be quickly learned even by beginners.
    \item Importing schemas from SDL facilitates migration from other languages.
    \item A type-safe client package allows developers to implement entire applications in one language or even query data for resolvers from other GraphQL servers.
\end{itemize}

\end{frame}

\begin{frame}\frametitle{Disadvantages}

\begin{itemize}
    \item Not all GrapHQL names can be represented with Haskell types and values.
    \item Since we use algebraic data types and records, we lose extensibility.
\end{itemize}

\end{frame}

\begin{frame}\frametitle{Conclusions}
    
    Since we use Haskell, we automatically gain laziness and purity that match the GraphQL specifications. This way, developers do not have to explicitly deal with concurrency and laziness and focus on business logic. 

    Alternative GraphQL Haskell libraries have a more extensible schema definition than our approach. However, implementing resolvers for these schemas is more straightforward in our library, as users receive more meaningful error messages and field values are irrelevant. Besides, our approach is more intuitive and requires less experience.
    
\end{frame}

\begin{frame}\frametitle{Outlook}

\begin{itemize}
    \li{Support interfaces}  Haskell does not provide interfaces, so we do not yet fully support them. One possible way is to support them with "type guards." 
    \li{Support custom collections} sets and non-empty collections cannot be modeled in GraphQL because it provides only list types for collections.  
    \li{Support directives} Since the GraphQL specification does not specify a particular implementation strategy for directives, it is up to each server library to provide a suitable API~\cite{schema-directives}.
    \li{Support input unions} GraphQL does not support input unions. While this function's value is essentially understood, its implementation is not cleared~\cite{gql-spec-input-unions}. 

    % \li{Type-Level Curried Arguments} flexible argument definitions for a small number of arguments than our current approach since the user does not have to define a new data type for each field. 
\end{itemize}

\end{frame}