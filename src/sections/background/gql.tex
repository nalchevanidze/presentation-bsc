\subsection{GraphQL}

\begin{frame}\frametitle{GraphQL}

  \begin{block}{What is GraphQL?}
    GraphQL is a query language for APIs. It was developed by Facebook and published in 2016~\cite{initial-analysis-of-gql,gql-spec}. 
  \end{block}

  \begin{block}{GraphQL as a current trend}
    It has a rich open-source ecosystem and the trust of companies in various industries such as GitHub, Netflix, PayPal, and others~\cite{morph-gql-1}.
  \end{block}

\begin{block}{Typed query language}
  It is a strongly typed language. The type system provides a solid contract between client and server, which eliminates possible invalid requests~\cite{real-time-sys-arc-based-on-gql}.
\end{block}

\end{frame}

\begin{frame}\frametitle{Advantages of GraphQL}

\begin{block}{Performance and efficiency improvement}
\begin{itemize}
  \item reduces numbers of API calls~\cite{migrating-to-gql}.
  \item reduction in response size~\cite{migrating-to-gql}.
\end{itemize}
\end{block}

\begin{block}{API versioning}
\begin{itemize}
  \item new fields added to a type do not result in client changes~\cite{migrating-to-gql}. 
\end{itemize}

\end{block}

\begin{block}{Facilitates rapid product development}
\begin{itemize}
  \item  Familiar syntax and semantics~\cite{rest-vs-gql-controlled-experiment}.
  \item real-time query validation and auto-completion~\cite{rest-vs-gql-controlled-experiment,migrating-to-gql}.
  \item Instant API exploration through introspection~\cite{migrating-to-gql}. 
\end{itemize}

\end{block}

\end{frame}

\begin{frame}\frametitle{GraphQL Language Components}

\begin{block}{Schema definition language (SDL)}
a domain-specific language, used by the server to define the schema~\cite{migrating-to-gql}.
\end{block}

\begin{block}{Query language}
a domain-specific language, used by the client to retrieve data from the server~\cite{initial-analysis-of-gql}. 
\end{block}

\begin{block}{Resolver functions}
Resolver functions return values from the underlying data structures based on client queries~\cite{migrating-to-gql}.
\end{block}

\end{frame}
