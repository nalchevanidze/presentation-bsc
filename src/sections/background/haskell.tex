\subsection{Haskell}
\begin{frame}\frametitle{Haskell}

Haskell is a non-strict, purely functional language with static typing and algebraic data types. Many developers think that Haskell programs look nice~\cite{history-of-haskell}.

\begin{block}{Haskell is lazy}
    % Laziness is a primary concern in Haskell's design. 
    Haskell is a non-strict semantic language; lazy evaluation is just a technique to implement it~\cite{history-of-haskell}.
\end{block}

\begin{block}{Haskell is pure}
The lazy evaluation requires a pure design since a function call can no longer guarantee the reliable execution of the side-effects~\cite{history-of-haskell}.
\end{block}

\end{frame}

\begin{frame}\frametitle{Algebraic Data Types}

% Algebraic data types and pattern matching are fundamental to most modern functional languages~\cite{trees-that-grow}. Using pattern matching against algebraic data types improves readability significantly~\cite{history-of-haskell}.

An algebraic data type is the sum of one or more alternatives, where each alternative is a product of zero or more fields (Haskell also allows a sum of zero alternatives, the so-called empty type)~\cite{history-of-haskell}. 
        
\importHS{maybe}{Algebraic Data Types~\cite{history-of-haskell}}

Once a data type is defined and compiled, its definition cannot be extended by adding new data constructors or new fields~\cite{trees-that-grow}.

\end{frame}

\begin{frame}\frametitle{Records }

The record system provides syntactic sugar for what might otherwise be written using ordinary, positional, algebraic data type declarations~\cite{lw-ext-records}. 

\importHSFragment{records}{0}{4}{Haskell Record}

Haskell record are not extensible. There are no operators for adding and removing fields in a record, we cannot reuse labels between different record types~\cite{poly-ext-records, hlist,lw-ext-records}. 

\end{frame}

\begin{frame}\frametitle{Record Values}

Records bring significant advantages and greatly simplify programming with data structures that have many components. Since the individual components are accessed by name (not by position), the fields' ordering is not crucial~\cite{lw-ext-records}.

\importHSFragment{records}{6}{10}{Haskell Record Values}
% For example, 
% the data type Deity declares a regular algebraic data type, but at the same time, constructor field accessors and modifiers. 

\end{frame}

\begin{frame}\frametitle{Monads}

Haskell abstracts side effects with monads. monads are formed by a type constructor \expr{M} and a pair of functions, \expr{return} and \expr{>>=}~\cite{history-of-haskell,essence-of-fp}. The type \expr{M a} is a computation that returns a value of type \expr{a} and possibly performs some side effects~\cite{history-of-haskell}. The purpose of \expr{return} and \expr{>>=} is to push a value into computation and to evaluate a computation, yielding a value~\cite{essence-of-fp}.

Since the monad allows the compiler to determine impure and pure operations, the Haskell language can use specific optimizations. However, the developer still has the freedom to define their execution order~\cite{history-of-haskell}.

\end{frame}

\begin{frame}\frametitle{Datatype-Generic Programming}
    
Datatype-generic programming enables writing single functions that address various cases and types.~\cite{derivable-type-classes}. 
It increases program reliability, reduces code duplication, and improves modularity while guaranteeing safety~\cite{datatype-generic-programming,optimizing-generics}.

% The standard examples are parsing,  serialization, or comparing of data types

\importHS{generics}{Generic Deriving}
    
The generic derivation represents all types and values by generic representation types. The user can convert any types to their generic representation, operate on them, and convert them back to their original types~\cite{optimizing-generics, ghc-generics}.

\end{frame}