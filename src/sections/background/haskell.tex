\subsection{Haskell}
\begin{frame}\frametitle{Haskell}

Haskell is a non-strict, purely functional language with static typing and algebraic data types. Many developers think that Haskell programs look nice~\cite{history-of-haskell}.

\begin{block}{Haskell is lazy}
    Laziness is a primary concern in the design of Haskell~\cite{history-of-haskell}.
    % Haskell is a non-strict semantic language; lazy evaluation is just a technique to implement it.
\end{block}

\begin{block}{Haskell is pure}
The lazy evaluation requires a pure design since a function call can no longer guarantee the reliable execution of the side-effects. However, Haskell permits side effects with monads~\cite{history-of-haskell}.
\end{block}

\end{frame}

\begin{frame}[allowframebreaks]\frametitle{Algebraic Data Types and Records}

An algebraic data type is the sum of one or more alternatives, where each alternative is a product of zero or more fields~\cite{history-of-haskell}. 
        
\importHS{maybe}{Algebraic Data Types~\cite{history-of-haskell}}

Records simplify programming with data structures because fields are accessed by name rather than position~\cite{lw-ext-records}.

\importHSFragment{records}{0}{4}{Haskell Records}

\importHSFragment{records}{6}{11}{Haskell Record Values}

Algebraic data types and records and are not extensible~\cite{lw-ext-records,trees-that-grow}. 

\end{frame}
