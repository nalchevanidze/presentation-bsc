%!TEX root = ../main.tex

\section{Motivation}

\begin{frame}{}
    \frametitle{Motivation}

    \footnotesize

    \begin{alertblock}{Aktuelle Herausforderungen}
        Heute werden riesige Mengen an Informationen über das Web ausgetauscht. Dies erhöht die \alert{Wartezeit} und die \alert{Komplexität} der Anwendungen. 
    \end{alertblock}

    \begin{alertblock}{Haskell und  GraphQL}
        Die Funktionssprache Haskell und GraphQL können diese Probleme verringern.
    \end{alertblock}

    \begin{block}{Fehlende Bibliothek}
        Es gibt einige Bibliotheken, die GraphQL in Haskell implementieren, aber entweder bieten sie keine ausreichende Typsicherheit oder sie bilden GraphQL in Haskell auf sehr komplizierte Weise ab. 
    \end{block}

\end{frame}

\setLayout{Zielsetzung}
\begin{frame}
    \frametitle{Zielsetzung}

    In dieser arbeit versuchen wir, eine Bibliothek bereitzustellen,
    die \alert{typensicherheit bietet} und dennoch \alert{einfach zu schreiben} ist. 
    dabei sollen folgende ziele verfolgt werden

        \footnotesize
        \begin{alertblock}{Sicherheit und Ausdrucksstärke}
            Reduzierung von Laufzeitfehlern und die Ermöglichung, komplexe Fälle im Code auszudrücken
        \end{alertblock}

        \begin{alertblock}{Flexibilität und Leistung} 
            Unterstützung verschiedener Nutzungskontexte trotz hoher Leistung. 
        \end{alertblock}

        \begin{alertblock}{Wartbarkeit und Benutzerfreundlichkeit} 
            Eine leicht verständliche (benutzbare) API und einen unkomplizierten Ansatz für die API-Definition bieten 
        \end{alertblock}
\end{frame}

%!TEX root = ../../main.tex

\section{Theoretische Grundlage}

\subsection{Haskell}
\setLayout{mainpoint}
\begin{frame}{}
    \frametitle{Haskell}
\end{frame}


\subsection{GraphQL}
\setLayout{mainpoint}
% \setBGColor{UHHRed}
\begin{frame}{}
    \frametitle{GraphQL}
\end{frame}


\section{Morpheus GraphQL}
\setBGColor{white}
\begin{frame}{Morpheus GraphQL}
    \begin{figure}
        \centering
        \includegraphics[width=1.1\textwidth]{assets/img/morpheus-graphql-bg.png}
    \end{figure}
\end{frame}

\section{Aktuelle Stand}

\begin{frame}{}
    \frametitle{Aktuelle Stand}

    \footnotesize

    \begin{alertblock}{woran ich arbeite?}
        \begin{enumerate}
            \item 
            Integration von  Curry colored Petri nets (CCPN)
            \item Definition von Abbildungsregeln 
        \end{enumerate}
    \end{alertblock}

    \begin{block}{welche Probleme auftreten?}
        kein Gewinn aus der Verwendung des CCPN mit GraphQL    
    \end{block}

\end{frame}