%!TEX root = ../main.tex

\section{Motivation}

\begin{frame}{}
    \frametitle{Motivation}

    Heute werden riesige Mengen an Informationen über das Web ausgetauscht. Dies erhöht die Komplexität der Anwendungen. Morpheus GraphQL versucht, diese Komplexität mit der Funktionssprache Haskell und GraphQL zu reduzieren, indem es Flexibilität und Typsicherheit bietet, indem es eine direkte api-Beschreibungsmethode mit regulären Haskell-Typen bereitstellt. 

\end{frame}

\setLayout{Zielsetzung}
\begin{frame}
    \frametitle{Zielsetzung}

    % Für die Arbeit werden die folgenden Ziele gesetzt.
    %  some important text will be \alert{highlighted} because it's important.

        \footnotesize
        \begin{alertblock}{Sicherheit und Ausdrucksstärke}
            Reduzierung von Laufzeitfehlern und die Ermöglichung, komplexe Fälle im Code auszudrücken
        \end{alertblock}

        \begin{alertblock}{Flexibilität und Leistung} 
            Unterstützung verschiedener Nutzungskontexte trotz hoher Leistung. 
        \end{alertblock}

        \begin{alertblock}{Wartbarkeit und Benutzerfreundlichkeit} 
            Eine leicht verständliche (benutzbare) API und einen unkomplizierten Ansatz für die API-Definition bieten 
        \end{alertblock}
\end{frame}

%!TEX root = ../../main.tex

\section{Theoretische Grundlage}

\subsection{Haskell}
\setLayout{mainpoint}
\begin{frame}{}
    \frametitle{Haskell}
\end{frame}


\subsection{GraphQL}
\setLayout{mainpoint}
% \setBGColor{UHHRed}
\begin{frame}{}
    \frametitle{GraphQL}
\end{frame}


\section{Morpheus GraphQL}
\setBGColor{white}
\begin{frame}{Morpheus GraphQL}
    \begin{figure}
        \centering
        \includegraphics[width=1.1\textwidth]{assets/img/morpheus-graphql-bg.png}
    \end{figure}
\end{frame}