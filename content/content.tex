
\section{Motivation}
%\subsection*{Motivation}

\begin{frame}
	\frametitle{Themenfindung}
	\begin{itemize}
		\item praktisches Thema im theoretischen Kontext
		\item Prozessgenerierung aus Petrinetzen\\
			als theoretischer Hintergrund
		\begin{itemize}
			\item nützlich zur Veranschaulichung in der Lehre
			\item zusätzliche Analysemöglichkeiten auf den Prozessen	
		\end{itemize}
		\item Praxis: Umsetzung automatisierter\\
		Prozessgenerierung in Renew
	\end{itemize}
\end{frame}

\section{Petrinetze}
\subsection{Modellierungstool}
\begin{frame}
	\frametitle{Herkunft}
	
	\begin{itemize}
		\item inhaltlich verwandt mit BPMN
		\item in den theoretischen Grundlagen\\
		verwandt mit Automatentheorie
		\item verschiedene Grade an Komplexität und Mächtigkeit
		\item Simulation von Nebenläufigkeit
		\item formaler mathematischer Unterbau
	\end{itemize}
\end{frame}


\subsection{Mathematischer Unterbau}

\begin{frame}
	\frametitle{Mathematischer Unterbau}
	
	\begin{itemize}
		\item Menge von Stellen S
		\item Menge von Transitionen T
		\item Flussrelation $(S x T) \cup (T x S)$
		\begin{itemize}
			\item[$\rightarrow$] Abbildung der Flussrelation nach $\mathbb{N}$
			\begin{itemize}
				\item[$\rightarrow$] für höhere Netze Abbildung auf Multimengen
			\end{itemize}
			\item 0 für nicht vorhandene Kanten
		\end{itemize}
		\item Markierung: Abbildung der Stellen nach $\mathbb{N}$
		\begin{itemize}
			\item[$\rightarrow$] für höhere Netze Abbildung auf Multimenge
			\item[$\rightarrow$] Startmarkierung
		\end{itemize}
%		\item Formel für Petrinetz: TODO
	\end{itemize}
\end{frame}


\section{Renew}
\subsection*{Renew}

\begin{frame}
	\frametitle{Renew}
	
	\begin{itemize}
		\item Renew:
		\begin{itemize}
			\item Tool zum Erstellen und Ausführen von Petrinetzen
			\item entwickelt von Arbeitsgruppe TGI ($\rightarrow$ ART) an diesem Fachbereich
			\item Analysetools können eingebunden werden
			\item meine Arbeit soll Renew erweitern
		\end{itemize}	
	\end{itemize}
\end{frame}

\section{Thema}
\subsection{Demo}
\begin{frame}
	\begin{center}
		\begin{Huge}
			\textbf{DEMO!}
		\end{Huge}
	\end{center}
\end{frame}

\subsection{Themenbeschreibung}

\begin{frame}
	\frametitle{Beschreibung des Themas}
	
	\begin{itemize}
		\item Generierung von Prozessen aus Petrinetzen
		\item Darstellung als Kausalnetze
		\item auf verschiedene Weisen darstellen
		\item Verwendbarkeit der entstandenen Netzen für Analyse
		\item Veranschaulichung von Prozessen (zu Lehrezwecken)
		\item Problemstellungen:
		\begin{itemize}
			\item ununterscheidbare Marken
			\item führende Transitionen in den Ausgangsnetzen
		\end{itemize}
	\end{itemize}
	
\end{frame}

\begin{frame}
	\frametitle{Abgrenzung}
	
	\begin{itemize}
		\item Voraussetzung: Ausgangsnetze sind endlich
		\item keine Analyse der entstandenen Netze
		\item keine vollständige Generierung aller Prozesse (nur ein Teil)
	\end{itemize}
	
\end{frame}

\section{Roadmap}

\subsection*{erreicht}
\begin{frame}
	\frametitle{Bisher erledigt}
	
	\begin{itemize}
		\item Einarbeitung und Arbeitsweise ca. 94h + 4h
		\begin{itemize}
			\item Renew, Redmine, git, ...
		\end{itemize}
		\item Erster Prototyp (1) ca. 30h + 11h
		\item Formatierung (2) ca. 5h
		\item Ausarbeitung ca. 87h
		\item Entkopplung (3) 23h + 8h
		\item Literaturrecherche ca. 46h
		\item Ausarbeitung ca. 49h
		\item interne Datenstruktur (4, 5, 6) 33h + 20h
		\item aktueller Sprint 2h + 5h
	\end{itemize}
\end{frame}

\begin{frame}
	\frametitle{inhaltlicher Stand}
	
	\begin{itemize}
%		\item letzte(r) Sprint(s) (siehe Demo):
%		\begin{itemize}
			\item Auslesen des Simulation-Log
			\item Generierung der Kausalnetze der entstandenen Prozesse
			\item P/T-Netze als Ausgangsnetze
			\item schlichte Formatierung
			\item knapp 60 Seiten Ausarbeitung
%		\end{itemize}
		
%		\item aktueller Sprint:
%		\begin{itemize}
%			\item Überarbeiten der Ausarbeitung
%			\item Einarbeitung von Quellen
%			\item Anmeldung der Arbeit
%		\end{itemize}
	\end{itemize}
\end{frame}

\subsection*{Plan}
\begin{frame}
	\frametitle{Nächste Schritte}
	
	\begin{itemize}
		\item Anmelden
		\item letzten Prototyp festlegen und bauen
		\item Aufräumen (Ausarbeitung und Code)
	\end{itemize}
\end{frame}

\section*{Links}

\begin{frame}
	\frametitle{Links}
	
	\begin{itemize}
		\item[Git] \url{https://git.informatik.uni-hamburg.de/tgi/theses/bsc-valentin-kroen}\\
			git@git.informatik.uni-hamburg.de:tgi/theses/bsc-valentin-kroen.git
		
		\item[Redmine] \url{https://tgi15.informatik.uni-hamburg.de/redmine/projects/bachelorarbeit-valentin-kroen}
	\end{itemize}
\end{frame}


\printbibliography

%%%%%%%%%%%%%%%%%%%%%%%%%%%%%%%%%%%%%%%%%%%%%%%%%%%%%%%%%%%%%%%%%%%%%%
%stops counter, such that backup frames aren´t included in the total framenumber
\appendix
\newcounter{finalframe}
\setcounter{finalframe}{\value{framenumber}}

\section{Geschichte dieser Präsi}
\begin{frame}
	\frametitle{Originalvorlage}
	
	\begin{itemize}
		\item Vorlage von WR
		\item[] \url{https://wr.informatik.uni-hamburg.de/teaching/organisatorische_hinweise}
		\item erstellt für Seminare, Praktika und Projekte
	\end{itemize}
\end{frame}

\begin{frame}
	\frametitle{Die Vorlage und ich}
	
	\begin{itemize}
		\item Seminar sowie Projekt BigData WiSe 17/18
		\begin{itemize}
			\item[$\rightarrow$] Erhalt der Vorlage
		\end{itemize}
		\item Abschlussarbeitenseminar, erste Version SoSe 18
		\item seither konstante Verbesserung
	\end{itemize}
\end{frame}


\section{Latex - Tipps und Tricks}
\subsection{Zusatzfolien}

\begin{frame}[fragile]
	\frametitle{HowTo Zusatzfolien}
	
	\begin{lstlisting}[language = tex]
		\appendix
		\newcounter{finalframe}
		\setcounter{finalframe}{\value{framenumber}}
		
		% add your extra slides here
		
		\setcounter{framenumber}{\value{finalframe}}		
	\end{lstlisting}
\end{frame}



%\begin{frame}
%	\frametitle{Code}
%	
%	\begin{lstlisting}[language = tex]
%		\begin{figure}
%			\only<1>{\includegraphics[height = 0.3\textheight]{images/Netze/Zyklus.eps}}%
%			\only<2>{\includegraphics[height = 0.3\textheight]{images/Netze/Zyklus_Start.eps}}%
%			\only<3>{\includegraphics[height = 0.3\textheight]{images/Netze/Zyklus_Step1.eps}}%
%			\only<4>{\includegraphics[height = 0.3\textheight]{images/Netze/Zyklus_Step2.eps}}%
%			\only<5>{\includegraphics[height = 0.3\textheight]{images/Netze/Zyklus_Step3.eps}}%
%			\only<6>{\includegraphics[height = 0.3\textheight]{images/Netze/Zyklus_Start.eps}}%
%			\only<7>{\includegraphics[height = 0.3\textheight]{images/Netze/Zyklus_Step1.eps}}%
%			\only<8>{\includegraphics[height = 0.3\textheight]{images/Netze/Zyklus_Step2.eps}}%
%			\only<9>{\includegraphics[height = 0.3\textheight]{images/Netze/Zyklus.eps}}
%			\caption{Einfacher Zyklus}
%			\label{Zyklus}
%		\end{figure}
%	\end{lstlisting}
%	
%\end{frame}


%\section{Vertiefung Petrinetze}
\section{Kausal-Netze}

\begin{frame}
	\frametitle{Kausal-Netze}
	
	\begin{itemize}
		\item eine Marke pro Stelle
		\item Transitionen dürfen beliebig verzweigen
		\item Stellen sind unverzweigt
		\item keine Zyklen
		\item Darstellung von Prozessen
	\end{itemize}
	
\end{frame}

\section{Ereignis-Netze}

\begin{frame}
	\frametitle{Ereignis-Netze}
	
	\begin{itemize}
		\item eine Marke pro Stelle
		\item Transitionen dürfen beliebig verzweigen
		\item Stellen dürfen nur vorwärts verzweigen
		\item keine Zyklen
		\item Darstellung von Branching-Prozessen
	\end{itemize}
	
\end{frame}

\section{BE-Netze}

\begin{frame}
	\frametitle{Bedingung-Ereignis-Netze}
	
	\begin{itemize}
		\item eine Marke pro Stelle
		\item Stellen und Transitionen dürfen beliebig verzweigen
		\item darf Zyklen enthalten
	\end{itemize}
	
\end{frame}



\section{PT-Netze}

\begin{frame}
	\frametitle{PT-Netze}
	
	\begin{itemize}
		\item beliebig viele Marken pro Stelle
		\item Stellen und Transitionen dürfen beliebig verzweigen
		\item darf Zyklen enthalten
		\item Kantengewichte
	\end{itemize}
	
\end{frame}



\section{gefärbte Netze}

\begin{frame}
	\frametitle{gefärbte Netze}
	
	\begin{itemize}
		\item beliebig viele Marken pro Stelle
		\item Stellen und Transitionen dürfen beliebig verzweigen
		\item darf Zyklen enthalten
		\item unterscheidbare Marken
		\item Erkennung des Markentyps
	\end{itemize}
	
\end{frame}

\section{Referenz-Netze}

\begin{frame}
	\frametitle{Referenz-Netze}
	
	\begin{itemize}
		\item beliebig viele Marken pro Stelle
		\item Stellen und Transitionen dürfen beliebig verzweigen
		\item darf Zyklen enthalten
		\item unterscheidbare Marken
		\item Erkennung des Markentyps
		\item Marken dürfen Objekte, insbesondere auch Netze sein
	\end{itemize}
	
\end{frame}
